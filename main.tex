\documentclass[12pt,a4paper]{article}
\usepackage{geometry}
\geometry{margin=2.5cm}
\usepackage{setspace}
\usepackage{graphicx}
% \usepackage{amsmath}
\usepackage{caption}
\usepackage{booktabs}
\usepackage{float}
\usepackage{xcolor}
\usepackage{tcolorbox}
\usepackage{titlesec}
\usepackage{hyperref}

% Etc
\usepackage{titling}
\usepackage{csquotes}

% math packages
\usepackage{enumitem}
\usepackage{amsmath,amssymb,amsthm}
\usepackage{algorithm}
\usepackage{algpseudocode}

\theoremstyle{remark}
\newtheorem*{solusi}{Solusi}

\setstretch{1.25}
\setlength{\parskip}{0.5em}
\setlength{\parindent}{0pt}

% --- Bahasa (Indonesia) ---
\usepackage[bahasa]{babel}

% --- Bibliografi (BibLaTeX) ---
\usepackage[backend=biber,style=ieee]{biblatex}
\addbibresource{refs.bib}

% % times new roman
% \usepackage{newtxtext,newtxmath}

\renewcommand\thesection{\Alph{section}}

% Format section menggunakan tcolorbox
\definecolor{headerbg}{RGB}{0,0,0}
\definecolor{headertext}{RGB}{255,255,255}


\newtcolorbox{sectionbox}{
  colback=headerbg,
  coltext=headertext,
  boxrule=0pt,
  arc=0pt,
  left=4pt,
  right=4pt,
  top=1pt,
  bottom=1pt,
  width=\dimexpr\textwidth\relax,
}

\titleformat{\section}[block]
  {\normalfont\bfseries}
  {}{0pt}
  {\begin{sectionbox}\thesection\ }[\end{sectionbox}]


% OVERRIDE Caption
\captionsetup[figure]{labelfont=bf, name=Gambar, justification=centering}
\captionsetup[table]{labelfont=bf, name=Tabel, justification=centering}


\begin{document}

\begin{center}
    \Large \textbf{BAB}\\[1em]
    \large \textbf{LOGIKA FUZZY}\\[2em]
\end{center}

\section{Tujuan Penjelasan Bab}
Setelah mengikuti penjelasan bab ini, pembaca diharapkan dapat:
\begin{enumerate}
    \item Memahami konsep dasar logika fuzzy serta perbedaannya dengan logika biner klasik.
    \item Menjelaskan struktur sistem fuzzy yang terdiri dari proses fuzzifikasi, basis aturan, mesin inferensi, dan defuzzifikasi.
    \item Mengidentifikasi jenis-jenis fungsi keanggotaan dan penerapannya dalam pemodelan ketidakpastian.
    \item Mengimplementasikan sistem inferensi fuzzy sederhana (misalnya metode Mamdani atau Sugeno) melalui contoh kasus nyata.
    \item Mengevaluasi kelebihan dan keterbatasan logika fuzzy dibandingkan dengan pendekatan deterministik maupun probabilistik.
\end{enumerate}


\section{Pendahuluan}

Sistem logika klasik, seperti logika proposisi (\textit{propositional logic}) dan logika predikat tingkat pertama (\textit{first-order logic}), setiap pernyataan hanya mengenal dua keadaan: benar (\textit{true}) atau salah (\textit{false}). Pendekatan biner seperti ini efektif untuk komputasi matematis dan digital, 
tetapi sering kali tidak cukup fleksibel untuk merepresentasikan kompleksitas dunia nyata. 
Dalam banyak kasus, fenomena nyata tidak sepenuhnya hitam atau putih, 
melainkan memiliki tingkat keabuan di antaranya.

Untuk mengatasi keterbatasan tersebut, muncul \textbf{logika fuzzy} 
sebagai metode yang memungkinkan nilai kebenaran berada di antara 0 dan 1. 
Dengan logika fuzzy, konsep seperti “suhu panas”, “kecepatan sedang”, 
atau “risiko rendah” dapat dimodelkan secara kuantitatif menggunakan 
fungsi keanggotaan (\textit{membership function}).

Pendekatan ini pertama kali diperkenalkan oleh Lotfi A. Zadeh pada tahun 1965, 
dan sejak itu telah menjadi dasar bagi berbagai sistem cerdas 
seperti kontrol suhu otomatis, sistem pendukung keputusan, dan klasifikasi data 
dalam bidang kecerdasan buatan.


\section{Konsep Utama}
Jelaskan konsep dasar dari topik yang dipilih.  
Misalnya: definisi, formulasi matematis, asumsi-asumsi dasar, dan hubungan konsep tersebut dengan cabang AI lain.  
Gunakan rumus bila relevan:
\[
    Y = \beta_0 + \beta_1 X + \epsilon
\]
Jelaskan arti tiap variabel, parameter, dan hubungannya dengan fenomena yang diamati.

\section{Penyelesaian Kasus}
Berikan contoh penerapan konsep di atas pada kasus nyata atau dataset sederhana.  
Contoh alur:
\begin{enumerate}
    \item Penjelasan data yang digunakan (sumber, variabel, jumlah observasi).
    \item Langkah-langkah penerapan metode.
    \item Perhitungan manual atau logika algoritmik.
    \item Interpretasi awal dari hasil.
\end{enumerate}

\section{Coding Penyelesaian Masalah}
Tuliskan kode yang digunakan untuk implementasi.  
Gunakan format \texttt{verbatim} atau \texttt{lstlisting} bila ingin menampilkan kode dengan format rapi, misalnya:

\begin{verbatim}
import numpy as np
from sklearn.linear_model import LinearRegression

X = np.array([[2],[3],[5],[7],[9],[11],[13],[15],[17],[19]])
y = np.array([50,55,65,70,80,85,90,92,94,98])

model = LinearRegression()
model.fit(X, y)

print("Intercept:", model.intercept_)
print("Slope:", model.coef_)
\end{verbatim}

\section{Hasil dan Interpretasinya}
Tampilkan hasil keluaran kode di atas (nilai koefisien, prediksi, error metrics, dsb.).  
Jelaskan interpretasinya dalam konteks permasalahan.  
Gunakan tabel atau gambar bila diperlukan.

\begin{table}[H]
\centering
\begin{tabular}{@{}lcc@{}}
\toprule
Variabel & Nilai Aktual & Prediksi Model \\ \midrule
2  & 50 & 52.1 \\
7  & 70 & 71.8 \\
15 & 92 & 91.3 \\ \bottomrule
\end{tabular}
\caption{Perbandingan Nilai Aktual dan Prediksi}
\end{table}

% \section{Referensi}
\printbibliography[title={Referensi}]

\end{document}
